% signature
\newcommand{\circuitsignature}{\Sigma}
\newcommand{\circuitsignaturevalues}{\mathcal{V}}
\newcommand{\circuitsignaturegates}{\mathcal{G}}
\newcommand{\circuitsignaturearity}{\#}

% values
\newcommand{\values}{\mathbf{V}}
\newcommand{\valuetuple}[1]{\tuple{\values}{#1}}
\newcommand{\valuestream}{\stream{\values}}
\newcommand{\valuetuplestream}[1]{\stream{(\valuetuple{#1})}}
\newcommand{\disconnected}{\bullet}
\newcommand{\shortcircuit}{\circ}

% logic gates
\newcommand{\nor}{{\downarrow}}
\newcommand{\xor}{\oplus}

% circuit morphisms
\newcommand{\fork}{{\prec}}
\newcommand{\join}{{\succ}}
\newcommand{\stub}{{\wr}}
\newcommand{\delay}{\delta}

% interpretation

\newcommand{\interpretation}{\mathcal{I}}

\NewDocumentCommand\valueinterpretation{o}{%
    \left\llbracket\IfValueTF{#1}{#1}{-}\right\rrbracket^{\mathbf{V}}
}
\NewDocumentCommand\gateinterpretation{o}{%
    \left\llbracket\IfValueTF{#1}{#1}{-}\right\rrbracket^{\mathbf{G}}
}

% belnap
\newcommand{\belnap}{B}
\newcommand{\belnapsignature}{\circuitsignature_{\belnap}}
\newcommand{\belnapvalues}{\circuitsignaturevalues_{\belnap}}
\newcommand{\belnapgates}{\circuitsignaturegates_{\belnap}}
\newcommand{\belnaparity}{\circuitsignaturearity_{\belnap}}

\newcommand{\belnapnone}{\mathsf{n}}
\newcommand{\belnapfalse}{\mathsf{f}}
\newcommand{\belnaptrue}{\mathsf{t}}
\newcommand{\belnapboth}{\mathsf{b}}

\newcommand{\belnaplattice}{\values_{\belnap}}

\newcommand{\belnapinterpretation}{\interpretation_{\belnap}}
\NewDocumentCommand\belnapvalueinterpretation{o}{%
    \valueinterpretation\IfValueT{#1}{[#1]}_{\belnap}
}
\NewDocumentCommand\belnapgateinterpretation{o}{%
    \gateinterpretation\IfValueT{#1}{[#1]}_{\belnap}
}

\NewDocumentCommand\circuittofunc{om}{%
    \left\llbracket\IfValueTF{#1}{#1}{-}\right\rrbracket^{\mathbf{C}}_{#2}
}
\NewDocumentCommand\circuittostream{om}{%
    \left\llbracket\IfValueTF{#1}{#1}{-}\right\rrbracket^{\mathbf{S}}_{#2}
}
\NewDocumentCommand\circuittomealy{om}{%
    \left[\IfValueTF{#1}{#1}{-}\right]_{#2}%
}

% extensional equivalence
\newcommand{\extequiv}[1]{\approx_{#1}}
\newcommand{\extequivi}{\extequiv{\interpretation}}

% combinational circuits
\newcommand{\ccirc}[1]{\mathbf{CCirc}_{#1}}
\newcommand{\ccircsigma}{\ccirc{\Sigma}}
% circuits with delay (temporal circuits)
\newcommand{\tcirc}[1]{\mathbf{TCirc}_{#1}}
\newcommand{\tcircsigma}{\tcirc{\Sigma}}
% circuits with feedback (sequential circuits)
\newcommand{\scirc}[1]{\mathbf{SCirc}_{#1}}
\newcommand{\scircsigma}{\scirc{\Sigma}}

% sequential circuits quotiented over some interpretation
\newcommand{\scircq}[2]{\mathbf{SCirc}_{#1,#2}}
\newcommand{\scircsigmai}{\scircq{\Sigma}{\interpretation}}
\newcommand{\scircsigmal}{\scircq{\Sigma}{\localequations}}
\newcommand{\scircsigmar}{\scircq{\Sigma}{\reductiveequations}}

% equational reasoning
\newcommand{\equations}[1]{\mathcal{E}_{#1}}
\newcommand{\commcomonoid}{CC}
\newcommand{\commmonoid}{CM}
\newcommand{\stmcequations}{\mathsf{STMC}}
\newcommand{\algebraicequations}{\mathcal{A}}
\newcommand{\bialgebraequations}{\mathcal{B}}
\newcommand{\cartesiannat}{\mathcal{N}}
\newcommand{\combinationalequations}{\mathcal{C}}
\newcommand{\mealyequations}{\mathcal{M}}
\newcommand{\delayequations}{\mathcal{D}}
\newcommand{\instantfeedbackequation}{\mathsf{IF}}
\newcommand{\waveformequation}{\mathsf{Wave}}
\newcommand{\bisimulationequation}{\mathsf{Bisim}}
\newcommand{\minimisationequation}{\mathsf{Min}}
\newcommand{\productiveequationsdelay}{\mathcal{P}^\diamond}
\newcommand{\productiveequations}{\mathcal{P}}
\newcommand{\reductiveequations}{\mathcal{R}}
\newcommand{\abstractionequations}{\mathcal{F}}
\newcommand{\localequations}{\mathcal{L}}

\newcommand{\instval}[1]{\mf{hd}(#1)}
\newcommand{\tail}[1]{\mf{tl}(#1)}
\newcommand{\wavef}[1]{\mathcal{W}(#1)}

\newcommand{\kstar}{\star}
\newcommand{\recur}[1]{(#1)^{\star}}

\newcommand{\sassg}{\Gamma}

\newcommand{\ginterp}[1]{\llbracket{#1}\rrbracket}
\NewDocumentCommand\eqaxioms{o}{%
    \IfValueTF{#1}{\stackrel{#1}{=}}{=}
}
\newcommand{\eqaxiomsc}{\eqaxioms{\mathcal{C}}}

\newcommand{\circuittransition}{\mathsf{T}}
\newcommand{\circuitreduce}{{\downarrow}}

\DeclareMathOperator{\andgate}{AND}
\DeclareMathOperator{\orgate}{OR}
\DeclareMathOperator{\notgate}{NOT}
\DeclareMathOperator{\mux}{MUX}