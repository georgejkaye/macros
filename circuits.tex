% signature
\newcommand{\circuitsignature}{\Sigma}
\newcommand{\circuitsignaturevalues}{\mathcal{V}}
\newcommand{\values}{\mathbf{V}}
\newcommand{\valuesk}{\mathbf{V}_k}
\newcommand{\circuitsignaturegates}{\mathcal{P}}
\newcommand{\circuitsignaturegatescomb}{\mathcal{C}}
\newcommand{\circuitsignaturegatesseq}{\mathcal{S}}
\newcommand{\circuitsignaturearity}{\dom}
\newcommand{\circuitsignaturecoarity}{\cod}
\newcommand{\circuitsignaturevaluecoarity}{\cod^\values}
\newcommand{\disconnected}{\bullet}

% combinational circuits
\newcommand{\ccirc}[1]{\mathbf{CCirc}_{#1}}
\newcommand{\ccircsigma}{\ccirc{\Sigma}}
\newcommand{\ccircsigmaplus}{\ccirc{\Sigma}^{+}}
% circuits with delay (temporal circuits)
\newcommand{\tcirc}[1]{\mathbf{TCirc}_{#1}}
\newcommand{\tcircsigma}{\tcirc{\Sigma}}
% circuits with feedback (sequential circuits)
\newcommand{\scirc}[1]{\mathbf{SCirc}_{#1}}
\newcommand{\scircsigma}{\scirc{\Sigma}}
\newcommand{\scircsigmaplus}{\scirc{\Sigma}^{+}}
\newcommand{\scircsigmap}{\scirc{\Sigma}^\mathsf{P}}

% interpretation
\newcommand{\interpretation}{\mathcal{I}}
\NewDocumentCommand\valueinterpretation{o}{%
    \left\llbracket\IfValueTF{#1}{#1}{-}\right\rrbracket^{\mathbf{V}}
}
\NewDocumentCommand\gateinterpretation{o}{%
    \left\llbracket\IfValueTF{#1}{#1}{-}\right\rrbracket
}

\NewDocumentCommand\circuittofunc{om}{%
    \left\llbracket\IfValueTF{#1}{#1}{-}\right\rrbracket^{\mathbf{C}}_{#2}
}
\NewDocumentCommand\circuittostream{om}{%
    \left\llbracket\IfValueTF{#1}{#1}{-}\right\rrbracket^{\mathbf{S}}_{#2}
}
\NewDocumentCommand\circuittostreami{o}{%
    \circuittostream[\IfValueTF{#1}{#1}{-}]{\interpretation}%
}
\NewDocumentCommand\circuittofunci{o}{%
    \circuittofunc[\IfValueTF{#1}{#1}{-}]{\interpretation}%
}
\NewDocumentCommand\circuittomealy{om}{%
    \left[\IfValueTF{#1}{#1}{-}\right]_{#2}%
}
\NewDocumentCommand\circuittomealyi{o}{%
    \circuittomealy[\IfValueTF{#1}{#1}{-}]{\interpretation}%
}

% extensional equivalence
\newcommand{\extequiv}[1]{\approx_{#1}}
\newcommand{\extequivi}{\extequiv{\interpretation}}

% sequential circuits quotiented over some interpretation
\newcommand{\scircq}[2]{\mathbf{SCirc}_{{#1} / {#2}}}
\newcommand{\scircsigmai}{\scircq{\Sigma}{\approx_{\interpretation}}}
\newcommand{\scircsigmae}{\scircq{\Sigma}{\mce_{\interpretation}}}
\newcommand{\scircsigmaobs}{\scircq{\Sigma}{\sim_{\interpretation}}}

% equational reasoning
\newcommand{\commcomonoid}{CC}
\newcommand{\commmonoid}{CM}
\newcommand{\stmcequations}{\mathsf{STMC}}
\newcommand{\algebraicequations}{\mathcal{A}}
\newcommand{\bialgebraequations}{\mathcal{B}}

% equation names
\newcommand{\combinationalequations}{\mcc}
\newcommand{\mealyequations}{\mathcal{M}}
\newcommand{\cartesianequations}{\mathcal{C}}
\newcommand{\normalisingequations}{\mathcal{N}_\interpretation}
\newcommand{\encodingequation}{\mathsf{Enc}}
\newcommand{\encodingequations}{\mathcal{H}}
\newcommand{\restrictionequation}{\mathsf{Res}}

\newcommand{\instantfeedbackeqn}{\mathsf{IF}}
\newcommand{\waveformeqn}{\mathsf{Wave}}
\newcommand{\bisimulationeqn}{\mathsf{Bisim}}
\newcommand{\minimisatiponequn}{\mathsf{Min}}

\newcommand{\cartnatcopyeqn}{\mathsf{NC}}
\newcommand{\cartnatdisceqn}{\mathsf{ND}}

\newcommand{\gateeqn}{\mathsf{Prim}_{\interpretation}}
\newcommand{\forkeqn}{\mathsf{Fork}}
\newcommand{\forkeqnshort}{\mathsf{F}}
\newcommand{\joineqn}{\mathsf{Join}}
\newcommand{\stubeqn}{\mathsf{Elim}}
\newcommand{\stubeqnshort}{\mathsf{E}}
\newcommand{\botregeqn}{\mathsf{Bot}}
\newcommand{\valregeqn}{\mathsf{Reg}}
\newcommand{\mealyeqn}{\mathsf{Mealy}}
\newcommand{\bundleeqn}{\mathsf{Bund}}
\newcommand{\unbundleeqn}{\mathsf{Split}}


\newcommand{\streamingeqn}{\mathsf{Str}}
\newcommand{\regforkeqn}{\mathsf{RFork}}
\newcommand{\regjoineqn}{\mathsf{RJoin}}
\newcommand{\regstubeqn}{\mathsf{RElim}}

\newcommand{\bottomdelayeqn}{\mathsf{BD}}

\newcommand{\gateforkeqn}{\mathsf{PF}}
\newcommand{\gatestubeqn}{\mathsf{PE}}
\newcommand{\joinstubeqn}{\mathsf{JE}}
\newcommand{\forkjoineqn}{\mathsf{FJ}}
\newcommand{\delayforkeqn}{\mathsf{DF}}
\newcommand{\delaydiscardeqn}{\mathsf{DD}}
\newcommand{\tracediscardeqn}{\mathsf{TD}}

\newcommand{\circuitbisim}{\mathsf{Bi}}
\newcommand{\cycleeqn}{\mathsf{Cycle}}
\newcommand{\closedfixpointeqn}{\mathsf{CFix}}
\newcommand{\unfoldeqn}{\mathsf{Unfold}}

\newcommand{\productiveequationsdelay}{\mathcal{P}^\diamond}
\newcommand{\productiveequations}{\mathcal{P}}
\newcommand{\reductiveequations}{\mathcal{R}}
\newcommand{\abstractionequations}{\mathcal{F}}
\newcommand{\localequations}{\mathcal{L}}

\newcommand{\inputcandidates}{\mathsf{in}}

\newcommand{\instval}[1]{\mathsf{hd}(#1)}
\newcommand{\tail}[1]{\mathsf{tl}(#1)}
\newcommand{\wavef}[1]{\mathcal{W}(#1)}

\newcommand{\kstar}{\star}
\newcommand{\recur}[1]{(#1)^{\star}}

\newcommand{\sassg}{\Gamma}

\newcommand{\ginterp}[1]{\llbracket{#1}\rrbracket}
\newcommand{\eqaxiomsc}{\eqaxioms{\mathcal{C}}}

\newcommand{\circuittransition}{\mathsf{T}}
\newcommand{\circuitreduce}{{\downarrow}}

\DeclareMathOperator{\andgate}{AND}
\DeclareMathOperator{\orgate}{OR}
\DeclareMathOperator{\notgate}{NOT}
\DeclareMathOperator{\mux}{MUX}
\DeclareMathOperator{\norgate}{NOR}

\newcommand{\andassoc}{\mathsf{AndAssoc}}
\newcommand{\andcomm}{\mathsf{AndComm}}
\newcommand{\anddistor}{\mathsf{AndDistOr}}
\newcommand{\anddistjoin}{\mathsf{ADJ}}
\newcommand{\orassoc}{\mathsf{OrAssoc}}
\newcommand{\orcomm}{\mathsf{OrComm}}
\newcommand{\ordistand}{\mathsf{OrDistAnd}}
\newcommand{\andabsoreqn}{\mathsf{AndAbsOr}}
\newcommand{\orabsandeqn}{\mathsf{OrAbsOr}}
\newcommand{\ordistjoin}{\mathsf{ODJ}}
\newcommand{\joindistand}{\mathsf{MDA}}
\newcommand{\joindistor}{\mathsf{MDO}}
\newcommand{\dneeqn}{\mathsf{DNE}}
\newcommand{\notjoin}{\mathsf{NM}}
\newcommand{\demorganand}{\mathsf{DM1}}
\newcommand{\demorganor}{\mathsf{DM2}}
\newcommand{\andidemeqn}{\mathsf{AndIdem}}
\newcommand{\oridemeqn}{\mathsf{OrIdem}}

\newcommand{\transitiontranslation}[1]{#1^{\leq}_0}
\newcommand{\outputtranslation}[1]{#1^{\leq}_1}

% booleans
\newcommand{\booleans}{\mathbf{B}}
\newcommand{\boolorder}{\leq_{\booleans}}
\newcommand{\band}{\land_{\booleans}}
\newcommand{\bor}{\lor_{\booleans}}
\newcommand{\bnot}{\neg_{\booleans}}
\newcommand{\nor}{{\downarrow}}
\newcommand{\xor}{\oplus}

% belnap
\newcommand{\belnaplattice}{\values_{\belnap}}
\newcommand{\belnapinterpretation}{\interpretation_{\belnap}}
\newcommand{\belnapfalse}{\mathsf{f}}
\newcommand{\belnaptrue}{\mathsf{t}}
\newcommand{\valuetuple}[1]{\tuple{\values}{#1}}
\newcommand{\valuestream}{\stream{\values}}
\newcommand{\valuetupleseq}[1]{(\valuetuple{#1})^\star}
\newcommand{\valuetuplestream}[1]{\stream{(\valuetuple{#1})}}
\newcommand{\belnap}{\mathsf{B}}
\newcommand{\belnapsignature}{\circuitsignature_{\belnap}}
\newcommand{\belnapvalues}{\values_{\belnap}}
\newcommand{\belnapgates}{\circuitsignaturegates_{\belnap}}
\newcommand{\belnaparity}{\circuitsignaturearity_{\belnap}}
\newcommand{\belnapcoarity}{\circuitsignaturecoarity_{\belnap}}
\NewDocumentCommand\belnapvalueinterpretation{o}{%
    \valueinterpretation\IfValueT{#1}{[#1]}_{\belnap}
}
\NewDocumentCommand\belnapgateinterpretation{o}{%
    \gateinterpretation\IfValueT{#1}{[#1]}_{\belnap}
}

% completeness
\newcommand{\belnaptobool}{\phi}
\newcommand{\belnaptofalse}{\belnaptobool_0}
\newcommand{\belnaptotrue}{\belnaptobool_1}

% explosion equations
\newcommand{\belnapexpeqn}{\mathsf{Exp}}
\newcommand{\joinandeqn}{\mathsf{BotJoinAnd}}
\newcommand{\joinoreqn}{\mathsf{BotJoinOr}}
\newcommand{\notforkeqn}{\mathsf{NotFork}}
\newcommand{\andforkeqn}{\mathsf{AndFork}}
\newcommand{\orforkeqn}{\mathsf{OrFork}}
\newcommand{\joinforkeqn}{\mathsf{JF}}
\newcommand{\botforkeqn}{\mathsf{BotFork}}
\newcommand{\andandidemeqn}{\mathsf{AndIdem}}
\newcommand{\ororidemeqn}{\mathsf{OrIdem}}
\newcommand{\andorddisteqn}{\mathsf{AndOrDist}}
\newcommand{\oranddisteqn}{\mathsf{OrAndDist}}
\newcommand{\botandeqn}{\mathsf{BotAnd}}
\newcommand{\botoreqn}{\mathsf{BotOr}}
\newcommand{\botnoteqn}{\mathsf{BotNot}}
\newcommand{\forkcommeqn}{\mathsf{ForkComm}}
\newcommand{\forkuniteqn}{\mathsf{ForkUnit}}
\newcommand{\forkassoceqn}{\mathsf{ForkAssoc}}

\newcommand{\equationdisplay}[3]{#1=#2\;(#3)}