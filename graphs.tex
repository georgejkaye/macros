% DEPENDENCIES: macros/category

% equivariance
\newcommand{\atoms}{\mathbb{A}}
\newcommand{\permaction}{\odot}
\newcommand{\rename}{\mathbin{\tau}}
\newcommand{\renameinv}{\rename^{-1}}

% hypergraph definition
\newcommand{\vertices}[1]{#1_\star}
\newcommand{\fbvertices}[1]{#1_{\mathsf{fb}}}
\newcommand{\edges}[3]{#1_{#2,#3}}
\newcommand{\sources}[1]{\mathsf{s}_{#1}}
\newcommand{\targets}[1]{\mathsf{t}_{#1}}
\newcommand{\labels}{\Lambda}

% categories of hypergraphs

\newcommand{\hyp}{\mathbf{Hyp}}
\newcommand{\labhyp}{\hyp_{\signature}}

% cospan bicategories of hypergraphs

\newcommand{\cspfihyp}{\mathsf{Csp}_{FI}(\labhyp)}
\newcommand{\macspfihyp}{\mathsf{MACsp}_{FI}(\labhyp)}
\newcommand{\smcspfihyp}{\mathsf{SMCsp}_{FI}(\labhyp)}
\newcommand{\lsmcspfihyp}{\mathsf{LSMCsp}_{FI}(\labhyp)}
\newcommand{\rsmcspfihyp}{\mathsf{RSMCsp}_{FI}(\labhyp)}

% hypergraph soundness
\NewDocumentCommand\termtohyp{om}{%
    \llbracket{\IfValueTF{#1}{#1}{-}}\rrbracket_{#2}
}

% hypergraph completeness
\NewDocumentCommand\term{ooo}{%
    \llangle\IfValueTF{#1}{#1}{-}\rrangle\IfValueTF{#3}{_{#3,#2}}{\IfValueT{#2}{_{#2}}}%
}
\NewDocumentCommand\stack{ooo}{%
    \mathsf{stack}\IfValueTF{#3}{_{#3,#2}}{\IfValueT{#2}{_{#2}}}\IfValueT{#1}{(#1)}%
}
\NewDocumentCommand\shuffle{ooo}{%
    \mathsf{shuffle}\IfValueTF{#3}{_{#3,#2}}{\IfValueT{#2}{_{#2}}}\IfValueT{#1}{(#1)}%
}

% graph rewriting
\newcommand{\deriv}[1]{\xRightarrow{#1}}
\newcommand{\rewrite}{\Rightarrow}
\newcommand{\grewrite}{\rightsquigarrow}
\newcommand{\rrule}[2]{\langle #1,#2 \rangle}

% equivalence maps
\newcommand{\homo}{\rightsquigarrow}
\newcommand{\hequiv}{\equiv_\mathsf{h}}
\newcommand{\vmap}[1]{{#1}_V}
\newcommand{\vmapm}[1]{{#1}_V}
\newcommand{\vmaps}[1]{{#1}_S}
\newcommand{\vmapss}[1]{{#1}_S^\mon}
\newcommand{\vmapt}[1]{{#1}_T}
\newcommand{\vmapts}[1]{{#1}_T^\mon}
\newcommand{\emap}[1]{{#1}_E}

\newcommand{\homeo}{\approx}

% legacy

\newcommand{\hypfun}[1]{\llbracket#1\rrbracket}
\newcommand{\hypfunsig}[2]{\tohyp{#1}_#2}
\newcommand{\inputs}[2][{}]{\mf{in}^{#1}(#2)}
\newcommand{\outputs}[2][{}]{\mf{out}^{#1}(#2)}
\newcommand{\ind}[1]{\mf{in}(#1)}
\newcommand{\oud}[1]{\mf{out}(#1)}
\newcommand{\inds}[1]{\mf{in}^\star(#1)}
\newcommand{\ouds}[1]{\mf{out}^\star(#1)}
\newcommand{\io}[1]{\mf{io}(#1)}
\newcommand{\elabels}{\labels^{\edges}}
\newcommand{\vlabels}{\labels^{\vertices}}
\newcommand{\shyp}{\mathbf{SHyp}}
\newcommand{\lhyp}{\mathbf{LHyp}}
\newcommand{\phyp}{\mathbf{PHyp}}
\newcommand{\ilhyp}{\mathbf{LHyp}^{\interface}}
\newcommand{\hypterm}{\mathbf{HypTerm}}
\newcommand{\lshyp}{\shyp_{\signature}}
\newcommand{\llhyp}{\lhyp_{\signature}}
\newcommand{\lilhyp}{\lhyp^{\interface}_{\signature}}
\newcommand{\lphyp}{\phyp_{\signature}}
\newcommand{\lhypterm}{\mathbf{HypTerm}_{\signature}}
\newcommand{\cspd}[1]{Csp_D(#1)}
\newcommand{\mcspd}[1]{MCsp_D(#1)}
\newcommand{\pmcspd}[1]{PMCsp_D(#1)}
\newcommand{\lmcspd}[1]{LMCsp_D(#1)}
\newcommand{\csplhyp}{\csp{\llhyp}}
\newcommand{\cspdlhyp}{\cspd{\llhyp}}
\newcommand{\mcspdlhyp}{\mcspd(\llhyp)}
\newcommand{\cspdphyp}{\cspd{\lphyp}}
\newcommand{\mcspdphyp}{\mcspd{\lphyp}}
\newcommand{\pmcspdphyp}{\pmcspd{\lphyp}}
\newcommand{\lmcspdphyp}{\lmcspd{\labhyp}}
\newcommand{\ghyp}{\mathbf{Hyp}_{\signature,\sorts}}
\newcommand{\gshyp}{\mathbf{SHyp}_{\signature,\sorts}}
\newcommand{\glhyp}{\mathbf{LHyp}_{\signature,\sorts}}
\newcommand{\gilhyp}{\mathbf{LHyp}^{\interface}_{\signature,\sorts}}
\newcommand{\ghypterm}{\mathbf{HypTerm}_{\signature,\sorts}}
\newcommand{\idedges}{\edges^{\id}}
\newcommand{\einput}{\alpha}
\newcommand{\eoutput}{\omega}
\newcommand{\interface}{\bullet}
\NewDocumentCommand\vt{oo}{%
    T\IfValueT{#2}{_#2}\IfValueT{#1}{[#1]}%
}
\NewDocumentCommand\vs{oo}{%
    S\IfValueT{#2}{_#2}\IfValueT{#1}{[#1]}%
}
\newcommand{\vconnsr}{\kappa}
\newcommand{\vconnsl}{\kappa^{-1}}
\newcommand{\htrim}[1]{\lceil #1 \rceil}
\newcommand{\hinter}[2]{\lfloor #1 \rfloor_#2}
\newcommand{\siface}[1]{\ulcorner #1 \urcorner}
\newcommand{\vtlabels}{\labels^{T}}
\newcommand{\vslabels}{\labels^{S}}
\newcommand{\vleft}{\lambda}                          % left 
\newcommand{\vright}{\rho}                            % right
\NewDocumentCommand\vtorder{o}{%
    \torder[\vt][\IfValueT{#1}{#1}]%
}
\NewDocumentCommand\vsorder{o}{%
    \torder[\vs][\IfValueT{#1}{#1}]%
}
\NewDocumentCommand\lhyper{o}{%
(\vs\IfValueT{#1}{_#1}, \vt\IfValueT{#1}{_#1}, \edges\IfValueT{#1}{_#1}, \esources\IfValueT{#1}{_#1}, \etargets\IfValueT{#1}{_#1}, \vconnsr\IfValueT{#1}{_#1})%
}
\NewDocumentCommand\llhyper{o}{%
(\vs\IfValueT{#1}{_#1}, \vt\IfValueT{#1}{_#1}, \edges\IfValueT{#1}{_#1}, \esources\IfValueT{#1}{_#1}, \etargets\IfValueT{#1}{_#1}, \vconnsr\IfValueT{#1}{_#1}, \labels\IfValueT{#1}{_#1})%
}
\NewDocumentCommand\lilhyper{o}{%
(\edges\IfValueT{#1}{_#1}, \vs\IfValueT{#1}{_#1}, \vt\IfValueT{#1}{_#1}, \vconnsr\IfValueT{#1}{_#1}, \labels\IfValueT{#1}{_#1})%
}
\NewDocumentCommand\lilhypere{o}{%
(\edges\IfValueT{#1}{_#1}, \edges\IfValueT{#1}{_#1}[\id], \vs\IfValueT{#1}{_#1}, \vt\IfValueT{#1}{_#1}, \vconnsr\IfValueT{#1}{_#1}, \labels\IfValueT{#1}{_#1})%
}
\newcommand{\hypers}{((\vt, \vtorder), (\vs, \vsorder), \edges, \inedges, \outedges, \vleft, \vright, \vconnsr, \labels)}
\newcommand{\includehypergraph}[2][0.35]{{\includesvg[scale=#1]{#2}}}